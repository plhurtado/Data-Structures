\begin{DoxyVersion}{\-Versión}
v0 
\end{DoxyVersion}
\begin{DoxyAuthor}{\-Autor}
\-Juan \-F. \-Huete
\end{DoxyAuthor}
\hypertarget{index_intro_sec}{}\section{\-Introducción}\label{index_intro_sec}
\-En esta practica se pretende avanzar en el uso de las estructuras de datos, para ello comenzaremos con el diseño de dos tipos de datos, el primero será un tipo capaz de guardar la información asociada a un meteorito, mientras que el segundo implicará el diseño de un contenedor básico, capaz de almacenar un gran volumen de meteoritos.\hypertarget{index_meteorito}{}\section{\-Conjunto de Datos de Meteoritos}\label{index_meteorito}
\-Este conjunto de datos contiene información sobre 45716 meteritos que han caido en la tierra, actualizado a 14 de \-Mayo de 2013, con los siguientes datos.

\begin{DoxyItemize}
\item nombre\-: por ejemplo \char`\"{}\-Andhara\char`\"{} o \char`\"{}\-Andover\char`\"{}, de tipo string. \item tipo\-: secuencia de códigos, separados por comas, que describen el tipo de meteorito, por ejemplo \char`\"{}\-Iron, I\-I\-A\-B\char`\"{}. \item masa\-: masa del meteorito en gramos. \item caido/encontrado\-: valor binario que indica si ha sido encontrado o sólo sabemos que ha caido (fell/found), \item año\-: año en que impactó \item coordena de latitud \item coordena de longitud\end{DoxyItemize}
\-Cuando alguno de los campos es desconocido, este está en blanco. \-La base de datos está en formato csv donde los distintos campos estan separados por punto y coma, por ejempo, las primeras líneas del fichero son\-:


\begin{DoxyCode}
name;recclass;mass;fall;year;reclat;reclong
Aachen;L5;21;Fell;1880;50,775;6,08333
Aarhus;H6;720;Fell;1951;56,18333;10,23333
Abee;EH4;107000;Fell;1952;54,21667;-113
Acapulco;Acapulcoite;1914;Fell;1976;16,88333;-99,9
\end{DoxyCode}
\hypertarget{index_met}{}\subsection{\-Meteorito}\label{index_met}
\-En nuestro caso, definiremos un tipo meteorito como un par, donde el campo first representa el nombre del meteorito (string) y el campo second será del tipo def\-Met, que define al resto de datos asociados que definen el meteorito.


\begin{DoxyCode}
typedef pair<nombreM,defM> meteorito; 
\end{DoxyCode}


\-La especificación del tipo, así como su implementación se realizará en los ficheros \hyperlink{meteorito_8h_source}{meteorito.\-h} y meteorito.\-hxx\hypertarget{index_dicc}{}\section{\-Diccionario}\label{index_dicc}
\-De forma abstracta un diccionario es un contenedor que permite almacenar un conjunto de pares de elementos, el primero será la clave que deber ser única y el segundo la definición. \-En nuestro caso, la clave será el nombre del meteorito y la defición contendrá valores de tipo \hyperlink{classdefM}{def\-M}. \-Como \-T\-D\-A diccionario, lo vamos a dotar de un conjunto restringido de métodos (inserción de elementos, consulta de un elemento por clave). \-Este diccionario \char`\"{}simulará\char`\"{} un diccionario de la stl, con algunas claras diferencias pues, entre otros, no estará dotado de la capacidad de iterar (recorrer) a través de sus elementos, que se hará en las siguientes prácticas.

\-Asociado al diccionario, tendremos los tipos tipos 
\begin{DoxyCode}
diccionario::entrada
diccionario::size_type
\end{DoxyCode}
 que permiten hacer referencia al par de elementos almacenados en cada una de las posiciones del diccionario y el número de elementos del mismo, respectivamente. \-El primer campo de una entrada, first, representa la clave y el segundo campo, second, representa la definición.\hypertarget{index_sec_tar}{}\section{\char`\"{}\-Se Entrega / Se Pide\char`\"{}}\label{index_sec_tar}
\hypertarget{index_ssEntrega}{}\subsection{\-Se entrega}\label{index_ssEntrega}
\-En esta práctica se entrega los fuentes necesarios para generar la documentación de este proyecto así como el código necesario para resolver este problema. \-En concreto los ficheros que se entregan son\-:

\begin{DoxyItemize}
\item documentacion.\-pdf \-Documentación de la práctica en pdf. \item dox\-\_\-diccionario \-Este fichero contiene el fichero de configuración de doxigen necesario para generar la documentación del proyecto (html y pdf). \-Para ello, basta con ejecutar desde la línea de comando 
\begin{DoxyCode}
 doxygen dox_diccionario
\end{DoxyCode}
 \-La documentación en html la podemos encontrar en el fichero ./html/index.html, para generar la documentación en latex es suficiente con hacer los siguientes pasos 
\begin{DoxyCode}
 cd latex
 make
\end{DoxyCode}
 como resultado tendremos el fichero refman.\-pdf que incluye toda la documentación generada.\end{DoxyItemize}
\begin{DoxyItemize}
\item \hyperlink{diccionario_8h_source}{diccionario.\-h} \-Especificación del \-T\-D\-A diccionario. \item \hyperlink{diccionario_8hxx_source}{diccionario.\-hxx} plantilla de fichero donde debemos implementar el diccionario. \item \hyperlink{meteorito_8h_source}{meteorito.\-h} \-Plantilla para la especificación del \-T\-D\-A meteorito\end{DoxyItemize}
\begin{DoxyItemize}
\item principal.\-cpp fichero donde se incluye el main del programa. \-En este caso, se toma como entrada el fichero de datos \char`\"{}meteorites\-\_\-all.\-csv\char`\"{} y se debe cargar en el diccionario.\end{DoxyItemize}
\hypertarget{index_ssPide}{}\subsection{\-Se Pide}\label{index_ssPide}
\-Se pide implementar el código asociado tanto para el \-T\-D\-A meteorito (en los ficheros meteoritos.\-h y meteorito.\-hxx) como el tipo de dato diccionario. \-En este último caso se considerarán dos posibles representaciones basadas en el tipo de dato vector de la stl, analizando la eficiencia de las mismas, teniendo en cuenta las operaciones de insercion y búsqueda. \-La primera implementación se entregará en un fichero denominado diccionario\-V1.\-hxx y la segunda en un fichero denominado diccionario\-V2.\-hxx

\-Para compilar con la primera implementación habrá que hacer

\begin{DoxyItemize}
\item g++ -\/\-D \-D\-I\-C\-C\-\_\-\-V1 -\/o corrector\-V1 principal.\-cpp\end{DoxyItemize}
\-Para compilar con la segunda implementación se tendrá que utilizar

\begin{DoxyItemize}
\item g++ -\/\-D \-D\-I\-C\-C\-\_\-\-V2 -\/o corrector\-V2 principal.\-cpp\end{DoxyItemize}
\-Por tanto, los alumnos deberán subir a la plataforma las dos implementaciones así como un análisis de la eficiencia de las mismas en los siguientes ficheros

\begin{DoxyItemize}
\item diccionario\-V1.\-hxx \item diccionario\-V2.\-hxx \item \-Analisis\-Compartivo.\-pdf \-Dicho análisis valorará por un lado el tiempo dedicado a la insercion de toda la colección de meteoritos en un diccionario y por otro el tiempo necesario para realizar una búsqueda de todos los meteoritos en el mismo.\end{DoxyItemize}
\hypertarget{index_fecha}{}\section{\-Fecha Límite de Entrega\-: Martes 28 de Octubre a las 23\-:59 horas.}\label{index_fecha}
\hypertarget{index_rep}{}\section{\-Representaciones}\label{index_rep}
\-El alumno deberá realizar dos implementaciones distintas del diccionario, utilizando como base el \-T\-D\-A vector de la \-S\-T\-L, en la primera de ellas los elementos se almacenarán sin tener en cuenta el valor de la clave mientras que en la segunda debemos garantizar que los elementos se encuentran ordenados por dicho valor.\hypertarget{index_primera}{}\section{\-Primera Representación\-:}\label{index_primera}
\hypertarget{index_fact_sec1}{}\subsection{\-Función de Abstracción \-:}\label{index_fact_sec1}
\-Función de \-Abstracción\-: \-A\-F\-: \-Rep =$>$ \-Abs

dado \-D=(vector$<$entrada$>$ dic) ==$>$ \-Diccionario \-Dic;

\-Un objeto abstracto, \-Dic, representando una colección de pares (clave,def) se instancia en la clase diccionario como un vector de entradas, definidas como diccionario\-::entrada. \-Dada una entrada, x, en \-D, x.\-first representa a una clave válida y x.\-second representa su definición.\hypertarget{index_inv_sec1}{}\subsection{\-Invariante de la Representación\-:}\label{index_inv_sec1}
\-Propiedades que debe cumplir cualquier objeto


\begin{DoxyCode}
Dic.size() == D.dic.size();



Para todo i, 0 <= i < D.dic.size() se cumple
        D.dic[i].first != "";
        D.dic[i].fisrt != D.dic[j].first, para todo j!=i.
\end{DoxyCode}
\hypertarget{index_segunda}{}\section{\-Segunda Representación\-:}\label{index_segunda}
\-En este caso, la representación que se utiliza es un vector ordenado de entradas, teniendo en cuenta el valor de la clave.\hypertarget{index_fact_sec2}{}\subsection{\-Función de Abstracción \-:}\label{index_fact_sec2}
\-Función de \-Abstracción\-: \-A\-F\-: \-Rep =$>$ \-Abs

dado \-D=(vector$<$entrada$>$ dic) ==$>$ \-Diccionario \-Dic;

\-Un objeto abstracto, \-Dic, representando una colección \-O\-R\-D\-E\-N\-A\-D\-A de pares (clave,def), se instancia en la clase diccionario como un vector de entradas, definidas como diccionario\-::entrada. \-Dada una entrada, x, en \-D, x.\-first representa a una clave válida y x.\-second representa su definición.(\hypertarget{index_inv_sec2}{}\subsection{\-Invariante de la Representación\-:}\label{index_inv_sec2}
\-Propiedades que debe cumplir cualquier objeto


\begin{DoxyCode}
Dic.size() == D.dic.size();

Para todo i, 0 <= i < D.dic.size() se cumple
        D.dic[i].first != "";
Para todo i, 0 <= i < D.dic.size()-1 se cumple
        D.dic[i].first< D.dic[i+1].first
\end{DoxyCode}
 